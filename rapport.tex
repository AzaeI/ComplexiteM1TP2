\documentclass[12pt]{report}
\usepackage[english]{babel}
\usepackage[utf8x]{inputenc}
\usepackage{amsmath}
\usepackage{graphicx}
\usepackage[colorinlistoftodos]{todonotes}
\usepackage[left=2cm,right=2cm,top=2.5cm,bottom=2cm]{geometry}
\usepackage[]{algorithm2e}
\usepackage{amsmath}
\usepackage{amssymb}

\begin{document}

%----------------------------------------------------------------------------------------
%	MAIN PAGE
%----------------------------------------------------------------------------------------
\begin{titlepage}
\vspace*{\fill}
\newcommand{\HRule}{\rule{\linewidth}{0.5mm}}

\center

\textsc{\LARGE Aix-Marseille Université}\\[1.5cm]
\textsc{\Large Compléxité}\\[0.5cm]

\HRule \\[0.4cm]
{ \huge \bfseries Le Solveur Minisat}\\[0.4cm] % Title of your document
\HRule \\[1.5cm]

\begin{minipage}{0.4\textwidth}
\begin{flushleft} \large
\emph{Autheur:}\\
Michaël \textsc{Gileta} \\
Yohan \textsc{Roux}
\end{flushleft}
\end{minipage}
~
\begin{minipage}{0.4\textwidth}
\begin{flushright} \large
\emph{Référent:} \\
Kévin \textsc{Perrot}
\end{flushright}
\end{minipage}\\[2cm]

{\large Vendredi 13 Octobre}\\[2cm]

\includegraphics[scale=0.5]{./images/logo.png}\\[1cm]

\vspace*{\fill}
\end{titlepage}

%----------------------------------------------------------------------------------------

\section*{Exercice 1}
\subsection*{Question a}
(A OR B OR NOT(C) OR D) \\
AND\\
(NOT(B) OR C)\\
AND\\
(NOT (A) OR NOT (D))

\subsection*{Question b}
\subsubsection*{Conversion format intermédiaire}
$( 1\| 2 \| -3 \| 4 )$\\
\& \\
$( -2 \| 3 )$\\
\& \\
$( -1 \| -4 )$
\subsubsection*{Conversion format Minisat}
p cnf 4 3\\
1 2 -3 4 0\\B
-2 3 0\\
-1 -4 0
\subsubsection*{Réponse Minisat}
La formule est satisfaisable.

\subsection*{Question c}
\subsubsection*{i.}
$\Phi = (\lnot t \rightarrow \lnot s) \rightarrow (((b \lor t) \rightarrow s) \land ((r \land m) \rightarrow (b \lor a))\land \lnot r)$ 
\begin{enumerate}
	\item $(t \lor \lnot s) \equiv (\lnot t \rightarrow \lnot s)$
	\item $(b \lor t) \rightarrow s \equiv (\lnot b \land \lnot t) \lor s \equiv (s\lor \lnot b) \land (s \lor \lnot t)$
	\item $(r \land m) \rightarrow (b \lor a) \equiv (\lnot r \lor \lnot m) \lor (b \lor a) \equiv (\lnot r \lor \lnot m \lor b \lor a)$
\end{enumerate}
$\Phi_2 = (t \lor \lnot s) \rightarrow ((s \lor \lnot b) \land (s \lor \lnot t) \land (\lnot r \lor \lnot m \lor b \lor a) \land \lnot r)$

$= (\lnot t \land s) \lor ((s \lor \lnot b) \land (s\lor \lnot t) \land (\lnot r \lor \lnot m \lor b \lor a) \land \lnot r)$

$= (\lnot t \lor s \lor \lnot b) \land (\lnot t \lor s) \land (\lnot t \lor \lnot r \lor \lnot m \lor b \lor a) \land (\lnot r \lor \lnot t) \land (s \lor \lnot b) \land (s \lor \lnot t) \land (s \lor \lnot r \lor \lnot m \lor b \lor a) \land (\lnot r \lor s)$

$= (\lnot t \lor s \lor \lnot b) \land (\lnot t \lor s) \land (\lnot t \lor \lnot r \lor \lnot m \lor b \lor a) \land (\lnot r \lor \lnot t) \land (s \lor \lnot b) \land (s \lor \lnot r \lor \lnot m \lor b \lor a) \land (s \lor \lnot r)$

\subsubsection*{ii.}
(NOT(T) OR S OR NOT(B) ) \\
AND \\
(NOT(T) OR S ) \\
AND \\
(NOT(T) OR NOT(R) OR NOT(M) OR B OR A ) \\
AND \\
(NOT(R) OR NOT(T) ) \\
AND \\
(S OR NOT(B) ) \\
AND \\
(S OR NOT(R) OR NOT(M) OR B OR A ) \\
AND \\
(S OR NOT(R) )

\subsubsection*{iii.}
Correspondance entre les numéros des variables et leur significations. \\
T = 1 \\
S = 2 \\
A = 3 \\
B = 4 \\
R = 5 \\
M = 6 \\
Lors de la première exécution, la formule est satisfaisable avec ce résultat : -1 -2 -3 -4 -5 -6 0
Ce qui correspond à l'affectation des valeurs correspondantes : \\
T = 0 \\
S = 0 \\
A = 0 \\
B = 0 \\
R = 0 \\
M = 0 \\
Pour avoir une autre solution il suffit d'ajouter le négatif de la solution trouvé en premier lieu en clause. \\
Nouvelle clause :  1 2 3 4 5 6 0 \\
On obtient bien une autre solution: -1 2 -3 -4 -5 -6 0 \\
T = 0 \\
S = 1 \\
A = 0 \\
B = 0 \\
R = 0 \\
M = 0 

\end{document}